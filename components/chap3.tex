\chapter{関連内容}

ここには,関連内容を書きます.
来年の方に,引き継ぐなど,より提案手法がわかりやすくなるように
手法で使用する技術の説明などを行います.

たとえば,群知能のアントコロニー最適化などに関する説明などです.


\section{アントコロニー最適化}


\subsection{群知能}

画像は以下のように入れることができます.
ラベルをつけると,他の場所\ref{fig:cute}から番号で参照でき,便利です.
(わざわざあとですべての図の番号を自分で変えなくて良いので)

tbpと指定するのが標準・学会デファクトらしいです.

\begin{figure}[tbp]
    \begin{center}
        \includegraphics[width=0.8\linewidth]{images/chapter3/cute.jpeg}
    \end{center}
    \caption{cute girl}
    \label{fig:cute}
\end{figure}

\section{LaTeX}

LaTeXでは,数式を利用することができます.

本文中なら\(x\),中央寄せなら\[y = x^2\]のようにすると良いです.

また,数式に番号をつけたい場合は\ref{eq:x3}のようにすると良いです.

\begin{equation}
    y = x^3
    \label{eq:x3}
\end{equation}

\section{箇条書き}

箇条書きも利用できます.

\begin{itemize}
    \item this
    \item is
\end{itemize}

数字付きも利用できます.
