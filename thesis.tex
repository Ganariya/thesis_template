% jsreportを使用します.
% hyperrefのためにdvipdfmxはここで読み込みます
\documentclass[a4paper,11pt,dvipdfmx]{jsreport}

% urlパッケージです URLを文書に埋め込むことができます.
\usepackage{url}
\usepackage{graphicx}

% Hを利用して強制的に表や図をその場に埋め込めますが,使わないほうが良いようです
\usepackage{here}

% ソースコードを埋め込むようです
% jlistingはstyファイルがディレクトリに入っています.
\usepackage{listings, jlisting}
\usepackage{ascmac,txfonts,txfonts}

% colorを使用できます
\usepackage{color}

% PDF内リンクがおこなえます. 読みやすくなります.
% レイアウトが崩れる場合は公式サイトを確認してエラーを直します.
\usepackage[setpagesize=false]{hyperref}
\usepackage{pxjahyper}
\usepackage{cite}

% アルゴリズムパッケージです
\usepackage{algorithm}
\usepackage{algorithmic}


% ソースコードを埋め込む場合の言語設定
\lstset{
	%プログラム言語(複数の言語に対応,C,C++も可)
 	language = Python,
 	%背景色と透過度
 	backgroundcolor={\color[gray]{.99}},
 	%枠外に行った時の自動改行
 	breaklines = true,
 	%自動改行後のインデント量(デフォルトでは20[pt])	
 	breakindent = 10pt,
 	%標準の書体
 	basicstyle = \ttfamily\scriptsize,
 	%コメントの書体
 	commentstyle = {\itshape \color[cmyk]{1,0.4,1,0}},
 	%関数名等の色の設定
 	classoffset = 0,
 	%キーワード(int, ifなど)の書体
 	keywordstyle = {\bfseries \color[cmyk]{0,1,0,0}},
 	%表示する文字の書体
 	stringstyle = {\ttfamily \color[rgb]{0,0,1}},
 	%枠 "t"は上に線を記載, "T"は上に二重線を記載
	%他オプション:leftline,topline,bottomline,lines,single,shadowbox
 	frame = TBrl,
 	%frameまでの間隔(行番号とプログラムの間)
 	framesep = 5pt,
 	%行番号の位置
 	numbers = left,
	%行番号の間隔
 	stepnumber = 1,
	%行番号の書体
 	numberstyle = \tiny,
	%タブの大きさ
 	tabsize = 4,
 	%キャプションの場所("tb"ならば上下両方に記載)
 	captionpos = t
}

% hlineの設定
\newcommand{\bhline}[1]{\noalign{\hrule height #1}}  
\newcommand{\bvline}[1]{\vrule width #1}  


% 文書の開始
\begin{document}

% タイトル
\begin{titlepage}
	\begin{center}

		{\Large 令和1年}

		\vspace{10truept}
		
		{\Large 卒業論文}

		\vspace*{130truept}

	
		{\huge 
			This is your \\
			title. \\
		} 

		\vspace{130truept}

		{\Large 指導教員}

		\vspace{10truept}

        {\Large who  教授}
        
        \vspace{3truept}

        {\Large who  教授}

		\vspace{70truept}

		{\Large xx大学 xx学部}

		\vspace{10truept}

		{\Large xx学科}

		\vspace{10truept}

		{\Large 7xxxxxxx yourname}		

	\end{center}
\end{titlepage}

% 概要
\begin{abstract}


ここはAbstractです.
この論文の概要をわかりやすく書きます.
偉い人はここしか読みません
\end{abstract}

% 目次 tocdepth 2で subsectionまでを目次に表示します
\setcounter{tocdepth}{2}
\tableofcontents
\clearpage

% 図目次
\listoffigures
\clearpage

% 表目次
\listoftables
\clearpage


% 本文
% inputを利用することでコンポーネントベースで卒論を書けます


\chapter{序論}

序論などを述べます.
各セクションのタイトルなどで何がやりたいかわかるといいようです.

\section{研究背景}

\section{研究目的}

\section{章構成}


\chapter{関連研究}

関連研究について示します.
論文紹介コーナーです.

論文の参照は\cite{bib:bird_swarm}のように行います.

\section{群知能}

\subsection{セクションより1小さい}

サブセクション

\subsubsection{サブセクションより1小さい}

サブサブセクション

\section{ソースコード}

ソースコードを載せる場合は以下のソースコード\ref{code:paken_day_3}のようにします.
Pythonのコードを掲載しています.
C++など他の言語や,行番号を消したい場合はthesis.texのlstsetの設定を変更してください.

\begin{lstlisting}[caption=paken, label=code:paken_day_3]
    N = int(input())

    S = [input().rstrip() for _ in range(5)]
    
    dc = {
        'W': 0,
        'B': 1,
        'R': 2,
    }
    
    dp = [[1e20] * 3 for _ in range(N + 1)]
    for i in range(3):
        dp[0][i] = 0
    
    # print(dp)
    
    for i in range(N):
    
        for j in range(3):
    
            for k in range(3):
    
                if j != k:
                    cost = 0
                    for y in range(5):
    
                        if S[y][i] == '#':
                            cost += 1
                        elif k != dc[S[y][i]]:
                            cost += 1
    
                    dp[i + 1][k] = min(dp[i + 1][k], dp[i][j] + cost)
    
    ans = min(dp[N][0], dp[N][1], dp[N][2])
    print(ans)
    
\end{lstlisting}
\chapter{関連内容}

ここには,関連内容を書きます.
来年の方に,引き継ぐなど,より提案手法がわかりやすくなるように
手法で使用する技術の説明などを行います.

たとえば,群知能のアントコロニー最適化などに関する説明などです.


\section{アントコロニー最適化}


\subsection{群知能}

画像は以下のように入れることができます.
ラベルをつけると,他の場所\ref{fig:cute}から番号で参照でき,便利です.
(わざわざあとですべての図の番号を自分で変えなくて良いので)

tbpと指定するのが標準・学会デファクトらしいです.

\begin{figure}[tbp]
    \begin{center}
        \includegraphics[width=0.8\linewidth]{images/chapter3/cute.jpeg}
    \end{center}
    \caption{cute girl}
    \label{fig:cute}
\end{figure}

\section{LaTeX}

LaTeXでは,数式を利用することができます.

本文中なら\(x\),中央寄せなら\[y = x^2\]のようにすると良いです.

また,数式に番号をつけたい場合は\ref{eq:x3}のようにすると良いです.

\begin{equation}
    y = x^3
    \label{eq:x3}
\end{equation}

\section{箇条書き}

箇条書きも利用できます.

\begin{itemize}
    \item this
    \item is
\end{itemize}

数字付きも利用できます.




\chapter{提案手法}
\label{sec:chap_4}

この章では,提案手法について記述します.
各章やセクションに関してもラベルを貼り,参照することができます.

\section{example}

たとえば,章を参照するときは\ref{sec:chap_4}のようにします.



\chapter{実験と評価}
\label{sec:experiment5}

この章では実験と評価について示します.

テーブルなどを利用する際は表\ref{tab:experiment_environment}のようにします.


\begin{table}[tbp]
    \caption{実験環境}
    \label{tab:experiment_environment}
    \begin{center}
        \begin{tabular}{cc} \bhline{1pt}
                OS      & CentOS:7                                       \\ \hline
                メモリ     & 4GB                                            \\
                CPU     & 4コア \\
                SSD     & 50GB                                           \\
                データセンター & 東京                                       \\ \hline
            \end{tabular}
    \end{center}
\end{table}


hlineを用いることで,表の上と下のみ黒い線にし,それ以外はほとんど線を引かないのが最近の学会の流行です.

tbpやcenter,ccなどはインターネットで詳しく書いてあります.

\chapter{議論}
\label{sec:discussion}


実験と評価をはみ出るような,全体的な考察や議論,どうすべきかなどを書きます.
いわゆる学会に出す論文ではかなり重要な部類に入る気がします.

\chapter{まとめ}

まとめを行います.

\chapter{Future Work}

これから行えることを書きます.

\clearpage

% 謝辞


\chapter*{謝辞}

感謝を述べます.

\*をつけると章に番号が付きません.


\clearpage

% 参考文献
% 参考文献とよばれるものです. 
% urlパッケージを用いると,きれいにURLを埋め込むことができます.
% 本文からcite{antsystem}のようにして参考論文を参照します.
\begin{thebibliography}{99}
    \bibitem{antsystem} Dorigo, Marco, and Gianni Di Caro. "Ant colony optimization: a new meta-heuristic." Proceedings of the 1999 congress on evolutionary computation-CEC99 (Cat. No. 99TH8406). Vol. 2. IEEE, 1999.
    \bibitem{bib:bird_swarm} EnWikipedia - "Swarm behaviour" \url{https://en.wikipedia.org/wiki/Swarm_behaviour}
\end{thebibliography}

% 付録
\appendix

\chapter{アペンディクス}

付録と呼ばれるものです.


\clearpage

\end{document}
